\documentclass[11pt,fleqn]{article}
\usepackage[fleqn]{amsmath}
\usepackage{xeCJK}
\usepackage{geometry}
\geometry{a4paper,total={170mm,257mm},left=20mm,top=20mm}
\allowdisplaybreaks[4]

\usepackage{fancyhdr}
\pagestyle{fancy}
\fancyhf{}
\fancyhead[CO, CE]{记叙文阅读}

\title{记叙文阅读}
\author{汪诺成}

\begin{document}

\begin{enumerate}
\item 表达方式:记叙、抒情、描写、说明、议论

\item 记叙的要素:时间、地点、人物,事情起因、经过、结果 \\
  概括内容 (记叙的要素、示例、要求)

\item
  $$
  \mbox{记叙的顺序}
  \begin{cases}
    \mbox{顺序} ( \mbox{时间推移、事件发展} ) \rightarrow \mbox{便于读者了解事情来龙去脉} \\
    \mbox{倒叙} 
    \begin{cases}
      \mbox{造成悬念,使叙事有波澜} \\
      \mbox{突现因果} \\
      \mbox{引起读者阅读兴趣}
    \end{cases} \\
    \mbox{插叙} \quad \frac{\mbox{写了}}{\mbox{(内容)}} ... \mbox{作用}
    \begin{cases}
      1. \mbox{解释补充、烘托主体事件} \\
      2. \mbox{展现人物性格} \\
      3. \mbox{内容更充实周密} \\
      4. \mbox{中心更鲜明}
    \end{cases}
  \end{cases}
  $$

\item
  $$
  \mbox{中心与情感}
  \begin{cases}
    1. \mbox{捕捉显性信息} \\
    2. \mbox{分析材料之间关系} \\
    3. \mbox{分析材料详略安排} \\
    4. \mbox{分析语言内在联系}
  \end{cases}
  $$

\item 文章线索: 人物、物品、感情、时间、事件

\item
  $$
  \mbox{结构层次}
  \begin{cases}
    \begin{aligned}
      \mbox{记事}    & : \mbox{时间先后,地点转换,发展阶段} \\
      \mbox{写人}     & : \mbox{成长阶段,地点变换,性格特征变化,感情变化} \\
      \mbox{写景状物} & : \mbox{空间变化,时间变化} \\
    \end{aligned}
  \end{cases}
  $$

\item
  $$
  \mbox{描写}
  \begin{cases}
    \mbox{环境描写} \quad \quad
    \begin{cases}
      1. \mbox{交代背景} \\
      2. \mbox{渲染气氛} \\
      3. \mbox{烘托形象} \\
      4. \mbox{推动情节发展} \\
      5. \mbox{深化主旨}
    \end{cases} \\
    \mbox{人物描写}
    \begin{cases}
      \mbox{肖像描写}
      \begin{cases}
        \mbox{表现内心} \\
        \mbox{突出形象} \\
        \mbox{表现时代特点}
      \end{cases} \\
      \mbox{语言描写}
      \begin{cases}
        \mbox{表现性格} \\
        \mbox{塑造形象} \\
        \mbox{推动情节发展} \\
        \mbox{交代事情经过} \\
        (\mbox{独自}:) \mbox{展示心理、抒发感情}
      \end{cases} \\
      \mbox{动作描写}
      \begin{cases}
        \mbox{表现性格} \\
        \mbox{揭示心理}
      \end{cases} \\
      \mbox{心理描写} : \mbox{直接表现情绪}
    \end{cases}
  \end{cases}
  $$

\item 分析语言
  $$
  \begin{aligned}
    & 1). \mbox{词语} \\
    & \quad \begin{aligned}
      & (1). \mbox{语境义}
      \begin{cases} \begin{aligned}
          \mbox{修辞义} &: \begin{aligned}[2]
            & \mbox{借代义} ( \mbox{相关性} ) & \mbox{反语} ( \mbox{讽刺性} ) \\
            & \mbox{比喻义} ( \mbox{相似性} ) & \mbox{比拟义} ( \mbox{形象性} )
          \end{aligned} \\
          \mbox{附加义} &: \mbox{情味感} \\
          \mbox{深层含义} &: \mbox{象征义} 
        \end{aligned} \end{cases} \\
      & (2). \mbox{词性}
      \quad \begin{aligned}
        \mbox{动词} & : \mbox{形容性,生动形象} \\
        \mbox{副词} & : \mbox{准确}
      \end{aligned} \\
      & (3). \mbox{感情色彩} : \mbox{褒义词、贬义词、中性词}
      \begin{cases}
        \mbox{表现人物情感} \\
        \mbox{表达作者语意倾向}
      \end{cases} \\
      & (4). \mbox{语言色彩}
      \quad \begin{cases}
        \mbox{口语} : \mbox{生动活泼,生活气息浓厚,多用于口语交际} \\
        \mbox{书面语} : \mbox{庄重文雅,多用于书面或严肃的场合}
      \end{cases} \\
      & (5). \mbox{叠词和谐音律 : 形象地描绘自然景色或人物特征,确切表
        达意象,给人美的享受} \\
    \end{aligned} \\
    & 2). TBD
  \end{aligned}
  $$

\end{enumerate}
\end{document}