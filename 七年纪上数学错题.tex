%%%  TeX-engine: xetex

\documentclass[UTF8]{ctexart}
\usepackage{ulem}
\usepackage{tikz}

\title{七年级上数学错题整理}

\newcommand\degree{^\circ}

\begin{document}
\maketitle
\tableofcontents

\section{重要考试}

\subsection{七年级期中} 

\begin{enumerate}
\item 如果关于x的不等式组$\begin{cases} m-4x>4 \\ x-\frac{11}{2}<3(x+\frac{1}{2}) \end{cases} $有且仅有三个奇数解,且关于x的方程式$\frac{2-mx}{2-x}-\frac{30}{x-12}=13$有非负数解,则符合条件的所有整数m的和是(     )

   A. 15
   B. 27
   C. 29
   D. 42

\item 若$ 2^m=a,32^n=b $ , m、n为正整数,则$2^{3m-10n}$=\underline{\quad\quad}

\item A是关于x的二次整式,且二次项系数为1,A与多项式$(x+2)$相乘后的结果为两项的多项式,则A=\underline{\quad}

\item 若关于x的方程$\frac{2x+m}{x-1}=3$的解为正整数,则m的取值范围是\underline{\quad\quad\quad}

\item 我们知道,同底数幂的乘法为:$a^ma^n=a^{m+n}(其中a \neq 0, m,n为正整数)$,类似地我们规定关于任意正整数m,n的一种新运算:$h(m+n)=h(m)h(n)$,请根据这种新运算填空:
   (1) 若$h(1)=\frac{2}{3}$,则$h{2}$=\underline{\quad\quad\quad};
   (2) 若$h(1)=k(k \neq 0)$,那么$h(n) \cdot h(2017)$=\underline{\quad\quad\quad}(用含n和k的代数式表示,其中n为正整数)。

\item 已知a、b、c、n是互不相等的正整数,且$\frac{1}{a}+\frac{1}{b}+\frac{1}{c}+\frac{1}{n}$也是整数,则n的最大值是\underline{\quad\quad\quad\quad}

\item 若一个自然数t能写成$t=x^2-y^2(x,y均为正整数,且x \neq y)$,则称t为“万象数”,x、y为t的一个万象分解,在t的所有万象分解中,若$\frac{x-y}{x+y}$最小,则称

\end{enumerate}

\subsection{七年级阶段性诊断2}

\begin{enumerate}

\item 计算: $(x+y)(-x-y)$=\underline{\quad\quad\quad}

\item 解方程: $\frac{2x+2}{x+3}-\frac{5}{7}=\frac{x}{x+3}$
   <br>
   <br>
   <br>

\item 2019年下半年受各种因素的影响,猪肉市场价格不断上升。据调查10月份猪肉的价格是9月份猪肉价格的1.25倍。小英妈妈用50元钱在10月份购得的猪肉比在9月份购得的猪肉少0.4斤,求2019年9月份的每斤猪肉价格
   <br>
   <br>
   <br>
   <br>

\item 如图,在直角三角形$ABC$中,$\angle B=90^\circ$,点M、N分别在边BA、BC上,且$BM=BN$。
   (1)画出直角三角形ABC关于直线MN堆成的三角形$A^`,B^`,C^`$;
   (2)如果$AB=a,BC=b,BM=x$ 用a、b、x的代数式分别表示三角形$AMA^`$的面积$S_1$和四边形$AA^`C^`C$的面积$S$,并简化。
   <br>
   <br>
   <br>
   <br>

 \end{enumerate}

\section{预习导航}

\subsection{16.1 二次根式预习导航}

\begin{enumerate}

\item 当x\underline{\quad\quad}时,$\frac{\sqrt{x-1}}{\sqrt{x}}$有意义;
   当x\underline{\quad\quad}时,$\frac{3\sqrt{x}}{1-\sqrt{x}}$有意义;
   当x\underline{\quad\quad}时,$\frac{1}{\sqrt{x}-2}$有意义;
   已知$\sqrt{a^2-2ab+b^2}=b-a$,则a\underline{\quad}b;
   当x满足\underline{\quad\quad}时,$\frac{\sqrt{1-3x}}{| x | - 3} $ 有意义。

\item 简化二次根式 $a\sqrt{-\frac{a+1}{a^2}}$的结果是\underline{\quad\quad}

    A. $\sqrt{-a-a}$
    B. $-\sqrt{-a-1}$
    C. $\sqrt{a+1}$
    D. $-\sqrt{-a+1}$
    
\item 化简 $x \sqrt{\frac{y}{x}} + y \sqrt{\frac{x}{y}}$

\end{enumerate}
   
\subsection{16.2(1) 最简二次根式预习导航}

\begin{enumerate}
    
\item 化简 $a\sqrt{\frac{1}{a^2}-\frac{1}{b^2}}$

\end{enumerate}

\subsection{16.2(2) 最简二次根式预习导航}

1. 下面说法正确的是

    A. 被开方数相同的二次根式是同类二次根式
    B. $\sqrt{8}$与$\sqrt{80}$是同类二次根式
    C. $\sqrt{2}$与$\sqrt{\frac{1}{50}}$不是同类二次根式
    D. 同类二次根式是根指数为2的根式
    
\subsection{16.3(2) 二次根式的运算预习导航(乘除)}

\begin{enumerate}
\item 计算:$3 \sqrt{5a} \cdot 2 \sqrt{10b}$=\underline{\quad\quad\quad\quad}
\item 使等式$\sqrt{(x+1)(x-1)}=\sqrt{x-1} \cdot \sqrt{x+1}$成立的条件是\underline{\quad\quad\quad}
\end{enumerate}
\subsection{16.3(3) 二次根式的混合运算预习导航}

\begin{enumerate}

\item 已知$x=\frac{\sqrt{3}-\sqrt{2}}{\sqrt{3}+\sqrt{2}},y=\frac{\sqrt{3}+\sqrt{2}}{\sqrt{3}-\sqrt{2}}$,则$x^2+y^2$的值为\underline{\quad\quad}

\item 化简$(\sqrt{\frac{x}{y}-2\sqrt{\frac{y}{x}}}) \cdot \sqrt{xy} \cdot \frac{x+y}{x-2y}$=\underline{\quad\quad\quad\quad}

\item 解答$[\frac{4}{(\sqrt{a}+\sqrt{b})(\sqrt{a}-\sqrt{b})} + \frac{\sqrt{a}+\sqrt{b}}{\sqrt{ab}(\sqrt{b}-\sqrt{a})}] \div \frac{\sqrt{a}-\sqrt{b}}{\sqrt{ab}}$,其中$a=4,b=4$

\end{enumerate}   

\section{新竹}

\subsection{16.1(1) 二次根式}

\begin{enumerate}

\item 如果$\sqrt{1-2a}$有意义,那么a的取值范围是\underline{\quad\quad}

\item 化简: $\sqrt{x^2-6x+9} + |1-x|(1<x<3)$

\end{enumerate}

\subsection{16.1(2) 二次根式}

\begin{enumerate}

\item 写出使下列等式成立的x的取值范围: $\sqrt{x^2(3-x)}=x \sqrt{3-x}$

\item 求下列各式成立时,x的取值范围: $\sqrt{\frac{2x-1}{3x+2}}=\frac{\sqrt{2x-1}}{\sqrt{3x+2}}$
   
\item 已知$\sqrt{a^3+3a^2}=-a\sqrt{a+3}$,求a的取值范围。

\item $\sqrt{\frac{3-y}{3+y}}=\frac{\sqrt{3-y}}{\sqrt{3+y}}$成立的条件是\underline{\quad\quad}

\item 已知实数满足$|1-x|=1+|x|$,化简$\sqrt{x^2(x-1)^2}$.

\end{enumerate}

\begin{tikzpicture}
  \draw (1,0) -- (0,0) -- (0,1);
\end{tikzpicture}

\end{document}
